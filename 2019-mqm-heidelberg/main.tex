\documentclass[final]{beamer}

\input{style.tex}

\usepackage[english]{babel}
\usepackage[utf8]{inputenc}
\usepackage{amsmath,amsthm,amssymb,latexsym}
\usepackage{multirow}
\usepackage[orientation=portrait,size=a0,scale=1.4]{beamerposter}
\usepackage{tikz}
\usetikzlibrary{arrows}

\usepackage[
  hyperref=false,
  url=false,
  abbreviate=true,
  maxnames=1,
  date=year,
  style=nature,
  backend=bibtex,
  doi=false
]{biblatex}
\AtEveryBibitem{\clearfield{note}}
\AtEveryBibitem{\clearfield{title}}
\bibliography{main.bib}


\title{
  Equation of motion coupled cluster theory using periodic boundary conditions
}
\author{\underline{Alejandro Gallo}, Felix Hummel and Andreas Gr\"uneis}
\def\correspondingAuthorEmail{alejandro.gallo@tuwien.ac.at}
\def\website{http://cqc.tuwien.ac.at}
\institute[]{
  Technical University of Vienna
}


\begin{document}
\nocite{*}
\begin{frame}[fragile]{}

  \begin{block}{\large Abstract}
    We present recent progress in the implementation of an equation of motion
    coupled cluster theory method using periodic boundary conditions and a plane
    wave basis set.  We seek to study electronic ground and excited states in
    Nitrogen Vacancy centers in diamond which have become a promising candidate
    for a bulk room temperature quantum information processing device. Technical
    details of the implementation and preliminary results will be presented.
  \end{block}
  \begin{columns}[t]
    \begin{column}{.49\linewidth}
      \begin{block}{\large Computational details}
        \begin{center}
\begin{tikzpicture}[style={
    blob/.style={
      node distance=4cm,
      minimum size=2cm,
    },
    cc4sstyle/.style={
      blob, circle,
      line width=5,
      inner color=green,
      outer color=red,
      draw,
    },
    ctfstyle/.style={
      blob, circle,
      line width=5,
      fill=green,
      draw=green!40!black,
    },
    vaspstyle/.style={
      blob, circle,
      line width=5,
      fill=red,
      draw=red!40!black,
    },
    myarrow/.style={->, shorten >=10pt, line width=3pt}
  }]
  \draw[]
    node[cc4sstyle] (cc4s) {\texttt{cc4s}}
    node[vaspstyle, below of=cc4s, left of=cc4s, yshift=-2cm] (vasp) {
      \texttt{VASP}
    }
    node[pos=0.5] {
    }
    node[ctfstyle, below of=cc4s, right of=cc4s, yshift=-2cm] (ctf) {
      \texttt{CTF}
    }
  ;
  \draw[myarrow] (vasp.north) |- (cc4s.west);
  \draw[myarrow]
    (ctf.north) |- (cc4s.east)
    node[pos=0.2, scale=0.7, right] {
      \texttt{
        T
        ["{\color{red}a}{\color{blue}b}{\color{green!50!black}i}{\color{brown}j}"] *
        V
        ["{\color{green!50!black}i}{\color{brown}j}{\color{red}a}{\color{blue}b}"]
      }
    }
  ;
\end{tikzpicture}
\end{center}

%\begin{itemize}
  %\item
    %Implementation in package \texttt{cc4s} (\textit{Coupled Cluster for
    %solids})
  %\item DFT and HF calculations performed with \texttt{VASP}
  %\item Plane waves basis set
  %\item Cyclops tensor framework (\texttt{CTF}) as a tensor contraction engine
%\end{itemize}

      \end{block}
      \begin{block}{\large Coupled-Cluster Ansatz}
          \[
    \hat{T} =
    \underbrace{
    \sum_{a,i}^{}
      {\color{blue}t^{a}_{i} }
      \hat{c}^{\dagger}_{a} \hat{c}_{i}
    +
    \sum_{a,b,i,j}
      {\color{blue}t^{ab}_{ij} }
      \hat{c}^{\dagger}_{a} \hat{c}^{\dagger}_{b}
      \hat{c}_{j}           \hat{c}_{i}
    }_{\mathrm{CCSD}}
    + \cdots
    =
    \tikz[baseline]\node{\includegraphics[scale=2]{figures/diagrams/t1.pdf}};
    +
    \tikz[baseline]\node{\includegraphics[scale=2]{figures/diagrams/t2.pdf}};
    +
    \cdots
  \]
\begin{itemize}
\item
  %
  %
  Exponential \textit{Ansatz} for dynamic correlation
  %
  \[
    \left | \mathsf{CC} \right \rangle  =
    e^{\hat{T}} \left | 0 \right \rangle
    =
    1 +
    \tikz[baseline]\node{\includegraphics[scale=2]{figures/diagrams/t1.pdf}};
    +
    \tikz[baseline]\node{\includegraphics[scale=2]{figures/diagrams/t2.pdf}};
    +
    \tikz[baseline]\node{\includegraphics[scale=2]{figures/diagrams/t1.pdf}};
    \tikz[baseline]\node{\includegraphics[scale=2]{figures/diagrams/t1.pdf}};
    +
    \tikz[baseline]\node{\includegraphics[scale=2]{figures/diagrams/t1.pdf}};
    \tikz[baseline]\node{\includegraphics[scale=2]{figures/diagrams/t2.pdf}};
    +
    \cdots
  \]
\item
  Projection-based iterative solution, setting matrix elements of
    $ \bar{\mathsf{H}} = e^{-T} H e^{T} $ equal to zero,
  \begin{align*}
    \bra{0}
      \hat{c}^{\dagger}_{i}
      \hat{c}_{a}
      \bar{\mathsf{H}}
    \ket{0}
    =
    \tikz[baseline, line width=2, scale=7]{
      \def\angle{15}
\def\waiLength{0.25}
\def\waiHeight{0.35}
% define main points
\draw[draw=none]
  (0,0)
    coordinate (l)
  -- ++ (\waiLength,0)
    coordinate (r)
    coordinate[pos=1] (center)
  (center) ++ (90 - \angle:\waiHeight)
    coordinate (topl)
  (center) ++ (90 + \angle:\waiHeight)
    coordinate (topr)
;
\draw[wiggly] (l) -- (r);
\draw[->-] (center) -- (topl);
\draw[-<-] (center) -- (topr);

    } &= 0
    \qquad \text{(singles)}
    \\
    \bra{0}
      \hat{c}^{\dagger}_{i}
      \hat{c}^{\dagger}_{j}
      \hat{c}_{b}
      \hat{c}_{a}
      \bar{\mathsf{H}}
    \ket{0}
    =
    \tikz[baseline, line width=2, scale=7]{
      % vim:ft=tex

\def\angle{15}
\def\wabijLength{0.4}
\def\wabijHeight{0.35}
% define points
\draw[draw=none]
  (0,0)
    coordinate (t2bottoml)
  ++ (\wabijLength,0)
    coordinate (t2bottomr)
  (t2bottoml) ++ (90 - \angle:\wabijHeight)
    coordinate (t2topparticlel)
  (t2bottoml) ++ (90 + \angle:\wabijHeight)
    coordinate (t2topholel)
  (t2bottomr) ++ (90 - \angle:\wabijHeight)
    coordinate (t2topparticler)
  (t2bottomr) ++ (90 + \angle:\wabijHeight)
    coordinate (t2topholer)
;

\draw[wiggly] (t2bottoml) -- (t2bottomr);

\draw[->-] (t2bottoml) -- (t2topparticlel);
\draw[->-] (t2bottomr) -- (t2topparticler);

\draw[-<-] (t2bottoml) -- (t2topholel);
\draw[-<-] (t2bottomr) -- (t2topholer);

    } &= 0
    \qquad \text{(doubles)}
  \end{align*}
\item
  Size consistency and extensivity
\end{itemize}


      \end{block}
      \begin{block}{\large Coupled-Cluster Ansatz}
        In one sentence:
\begin{center}
  \textit{
  Perform \emph{CI} on the similarity-transformed Hamiltonian $\bar{\mathsf{H}}$
  instead of $\hat H$.
}
\end{center}
\begin{itemize}
  \item
    Linear \textit{Ansatz} on top of ground state coupled-cluster.
    \begin{align*}
      \ket{r_i} &= \hat{R} ^{(i)} e^{\hat{T}} \ket{0}
      \\
      \bra{l_i} &= \bra{0} e^{-\hat{T}} \hat{L}^{(i)}
    \end{align*}
    where
      \begin{align*}
        \hat{R} ^{(i)} &= r _{0}
         + r ^{a} _{i} \hat{c} ^{\dagger} _{a} \hat{c} _{i}
         + r ^{ab} _{ij} \hat{c} ^{\dagger} _{a} \hat{c} ^{\dagger} _{b}
                \hat{c} _{j} \hat{c} _{i}
         + \ldots
        =
        \tikz[baseline]\node{\includegraphics[scale=2]{figures/diagrams/r1.pdf}};
        +
        \tikz[baseline]\node{\includegraphics[scale=2]{figures/diagrams/r2.pdf}};
        +
        \cdots
        \\
        \hat{L} ^{(i)} &= l _{0}
                       + l ^{i} _{a} \hat{c} _{i}^{\dagger} \hat{c} _{a}
                       + l ^{ij} _{ab}
                          \hat{c} _{j} ^{\dagger} \hat{c} _{i}^{\dagger}
                          \hat{c} _{a} \hat{c} _{b}
                       + \ldots
        =
        \tikz[baseline]\node{\includegraphics[scale=2,angle=180]{figures/diagrams/r1.pdf}};
        +
        \tikz[baseline]\node{\includegraphics[scale=2,angle=180]{figures/diagrams/r2.pdf}};
        +
        \cdots
      \end{align*}

  \item
    Non-hermitian diagonalization problem of
      $ \bar{\mathsf{H}} = e^{-\hat{T}} \hat{H} e^{\hat{T}}$

  \item
    Left and right eigenvectors $ \Rightarrow $ accessibility to calculate
    properties.

\end{itemize}

\textbf{Davidson-based diagonalization procedure}
\begin{itemize}

  \item
    Get initial \textit{right} basis
    $ B = \left \{ b_{1}, \ldots, b_{k} \right \} $
    to obtain $ k $ excited states

  \item
    Compute reduced matrix and diagonalize
    $
      h_{ij} =
      \bra{b_{i}} \bar{\mathsf{H}} \ket{b_{j}}
    $

  \item
    Get $ k $ lowest eigenvalues and eigenvectors and compute corrections
    for them

  \item
    If the corrections are significant, add them to the basis

  \item
    Basis refreshment: at some iterations, the basis gets collapsed
    to have only $ k $ vectors

\end{itemize}

      \end{block}
    \end{column}

    % -------------------------------------------------------------

    \begin{column}{.49\linewidth}

      \begin{block}{\large Color centers in alkali halides}
        \includegraphics[width=0.8\textwidth]{figures/fcenters/plot.pdf}
      \end{block}

      \begin{block}{\large Molecule benchmark example: Formaldehyde (CH$_2$O)}
          %\begin{columns}[t]
            %\begin{column}{.2\linewidth}
              %\begin{itemize}
                %\item
                  %Calculation with pseudized-gaussians \cite{10.10631.4961301}
              %\end{itemize}
            %\end{column}
            %\begin{column}{.8\linewidth}
            %\end{column}
          %\end{columns}
          \includegraphics[width=\textwidth]{figures/formaldehyde/plot.pdf}
          \begin{itemize}
            \item
              Calculation with pseudized-gaussians
                (see \cite{10.10631.4961301}).
            \item
              Singlet and triplet states retrieved.
          \end{itemize}
      \end{block}

      \begin{block}{\large References \& Acknowledgements}
        \printbibliography
      \end{block}

    \end{column}

  \end{columns}

\end{frame}
\end{document}
