In one sentence:
\begin{center}
  \textit{
  Perform \emph{CI} on the similarity-transformed Hamiltonian $\bar{\mathsf{H}}$
  instead of $\hat H$.
}
\end{center}
\begin{itemize}
  \item
    Linear \textit{Ansatz} on top of ground state coupled-cluster.
    \begin{align*}
      \ket{r_i} &= \hat{R} ^{(i)} e^{\hat{T}} \ket{0}
      \\
      \bra{l_i} &= \bra{0} e^{-\hat{T}} \hat{L}^{(i)}
    \end{align*}
    where
      \begin{align*}
        \hat{R} ^{(i)} &= r _{0}
         + r ^{a} _{i} \hat{c} ^{\dagger} _{a} \hat{c} _{i}
         + r ^{ab} _{ij} \hat{c} ^{\dagger} _{a} \hat{c} ^{\dagger} _{b}
                \hat{c} _{j} \hat{c} _{i}
         + \ldots
        =
        \tikz[baseline]\node{\includegraphics[scale=2]{figures/diagrams/r1.pdf}};
        +
        \tikz[baseline]\node{\includegraphics[scale=2]{figures/diagrams/r2.pdf}};
        +
        \cdots
        \\
        \hat{L} ^{(i)} &= l _{0}
                       + l ^{i} _{a} \hat{c} _{i}^{\dagger} \hat{c} _{a}
                       + l ^{ij} _{ab}
                          \hat{c} _{j} ^{\dagger} \hat{c} _{i}^{\dagger}
                          \hat{c} _{a} \hat{c} _{b}
                       + \ldots
        =
        \tikz[baseline]\node{\includegraphics[scale=2,angle=180]{figures/diagrams/r1.pdf}};
        +
        \tikz[baseline]\node{\includegraphics[scale=2,angle=180]{figures/diagrams/r2.pdf}};
        +
        \cdots
      \end{align*}

  \item
    Non-hermitian diagonalization problem of
      $ \bar{\mathsf{H}} = e^{-\hat{T}} \hat{H} e^{\hat{T}}$

  \item
    Left and right eigenvectors $ \Rightarrow $ accessibility to calculate
    properties.

\end{itemize}

\textbf{Davidson-based diagonalization procedure}
\begin{itemize}

  \item
    Get initial \textit{right} basis
    $ B = \left \{ b_{1}, \ldots, b_{k} \right \} $
    to obtain $ k $ excited states

  \item
    Compute reduced matrix and diagonalize
    $
      h_{ij} =
      \bra{b_{i}} \bar{\mathsf{H}} \ket{b_{j}}
    $

  \item
    Get $ k $ lowest eigenvalues and eigenvectors and compute corrections
    for them

  \item
    If the corrections are significant, add them to the basis

  \item
    Basis refreshment: at some iterations, the basis gets collapsed
    to have only $ k $ vectors

\end{itemize}
